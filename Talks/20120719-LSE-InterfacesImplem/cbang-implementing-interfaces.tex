%% Implementing interfaces in C!
%% Marwan Burelle - LSE - july 2012

\documentclass{beamer}

\mode<presentation>
{
  \usetheme{LSEweek2012}
  \setbeamercovered{dynamic}
}

\usepackage{moreverb}
\usepackage[english]{babel}
\usepackage[utf8]{inputenc}
\usepackage{txfonts}
\usepackage{microtype}
\usepackage[T1]{fontenc}
\usepackage{graphicx}
\usepackage{calrsfs}
\usepackage{alltt}
\usepackage{listings}
\usepackage{tikz}
\usetikzlibrary{arrows,snakes,backgrounds,shapes}
\usetikzlibrary{trees}

\newcommand{\cbang}{{\bf C!}}

\title{\cbang{}}
\subtitle{Implementing Interfaces}

\author{Marwan Burelle}

\institute{
  marwan.burelle@lse.epita.fr\\
  \url{http://www.lse.epita.fr/}
}

\date{}

\definecolor{darkblgr}{rgb}{0.7,0.7,0.75}
\setbeamercolor{bSection}{fg=black,bg=darkblgr}

\AtBeginSection[]
{
  \begin{frame}<beamer>
    \frametitle{\insertsection}
    \begin{beamerboxesrounded}[lower=bSection]{}
      \begin{center}
        \huge{\bf \insertsection}
      \end{center}
    \end{beamerboxesrounded}
  \end{frame}
}

\AtBeginSubsection[]
{
  \begin{frame}<beamer>
    \frametitle{\insertsubsection}
    \begin{beamerboxesrounded}[lower=bSection]{}
      \begin{center}
        \textbf{\insertsubsection}
      \end{center}
    \end{beamerboxesrounded}
  \end{frame}
}

\definecolorset{cmyk}{}{}%
 {olive,0,0,1,.5}

\definecolor{warnbg}{rgb}{0.3,0,0.1}
\setbeamercolor{warnbl}{fg=white,bg=warnbg}
\newenvironment{BoxedWarning}
{\begin{beamerboxesrounded}[lower=warnbl]{}}{\end{beamerboxesrounded}}

\newcommand{\DefBackButton}[2]{\hfil\hyperlink{#1<#2>}{\beamerreturnbutton{back}}}

\setbeamercolor{CamlRep}{fg=olive}
\newcommand{\CamlRep}[1]{{\usebeamercolor[fg]{CamlRep}\emph{\texttt{#1}}}}

\setbeamercolor{PosYes}{fg=green}
\newcommand{\yes}[1]{{\usebeamercolor[fg]{PosYes} #1}}

\begin{document}

\definecolor{orange}{rgb}{1,0.3,0}
\definecolor{kword}{rgb}{0,0.4,0.2}
\definecolor{dblue}{rgb}{0,0,0.7}
\definecolor{strcolor}{rgb}{0.5,0.2,0}
\definecolor{colorlabel}{rgb}{0.7,0,0}

\lstset{
  language=C,
  basicstyle=\ttfamily,
  commentstyle=\color{orange},
  keywordstyle=\color{kword}\bfseries,
  stringstyle=\color{strcolor},
  showstringspaces=false,
  identifierstyle=,
  emph={CAST},
  emphstyle=\color{red},
  emph={[2]tryagain},
  emphstyle={[2]\color{colorlabel}},
}

\begin{frame}
  \titlepage{}
\end{frame}

\begin{frame}<beamer>
  \frametitle{Outline}
  \tableofcontents
\end{frame}

\section{Introduction}

\begin{frame}
  \frametitle{Quick Overview of \cbang{}}
  \begin{itemize}
  \item System oriented programming language
  \item Variant of C with a rationnalized syntax.
  \item Syntactic Sugar.
  \item Object Oriented extensions.
  \item \cbang{} is a \emph{compiler-to-compiler} producing standard C code
  \end{itemize}
\end{frame}

\begin{frame}
  \frametitle{Motivation}
  \begin{itemize}
  \item \cbang{} was designed for kernel programming
  \item The aim of the project is to provide modern extensions to C (like
    \emph{OOP}) without breaking the aspects that make C a good kernel
    programming language.
  \item \cbang{} has no run-time depencies: make it easier to integrate in a
    non-standard environment.
  \item We try to minimize the amount of \emph{hidden code}: almost all
    expensive operations are explicit (no hidden object copy for example.)
  \item Since we produce C code, it is quite easy to integrate \cbang{} pieces
    of code in existing C projects.
  \end{itemize}
\end{frame}

\section{Objects in \cbang{}}

\begin{frame}
  \frametitle{Object Model}
  \begin{itemize}
  \item The \emph{OOP} model provides:
    \begin{itemize}
    \item Class definitions with virtual methods (true methods)
    \item Simple inheritance and subtyping based polymorphism
    \item Interfaces and abstract methods
    \item Method overriding
    \item No overloading
    \item No visibility control
    \end{itemize}
  \item In addition, objects instances follow some rules:
    \begin{itemize}
    \item Objects are always pointer
    \item You can define local object on the stack
    \item You have to provide your own allocation scheme
    \end{itemize}
  \end{itemize}
\end{frame}

\begin{frame}
  \frametitle{Interfaces and Abstract Methods}
  \begin{itemize}
  \item You can define interfaces: pure abstract classes with no state and no
    implementation for methods.
  \item Interfaces can inherit other interfaces and unlike classes
    inheritance, you can have multiple inheritance.
  \item Class can explicitely \emph{support} interfaces (no structural typing)
  \item When implementing an interface you (as usual) must provides an
    implementation for each methods in the interface.
  \item You can have abstract methods in class: methods with no implementation
  \item Classes with abstract methods can't be instantiated
  \end{itemize}
\end{frame}

\section{The Inner Side}

\subsection{Objects}

\begin{frame}
  \frametitle{Objects in C}
  \begin{itemize}
  \item While C does not provides objects, we can use pure C constructions to
    represent them.
  \item We only need structure and function pointers.
  \item Thus, \cbang{} \emph{OOP} extensions can be encoded as (almost) pure
    syntactic sugar.
  \item Since we produce C code, we take advantage of automatic offset
    computation in structure.
  \item Using structures rather than annonymous table generate more readable
    code.
  \end{itemize}
\end{frame}

\begin{frame}[fragile]
  \frametitle{Basic Objects}
  \begin{itemize}
  \item An object is the association of some data (attributes) and operations
    on this data (methods)
  \item We just put function pointers in a structure.
  \end{itemize}
  \begin{BoxedExample}
    \footnotesize
    \begin{columns}
      \begin{column}{3.5cm}
\begin{lstlisting}
class obj {
 get() : int;
 set(int x) : void;
 int content;
}
\end{lstlisting}
      \end{column}
      \begin{column}{6cm}
\begin{lstlisting}
struct s_obj {
 int (*get)(struct s_obj*);
 void (*set)(struct s_obj*,int);
 int content;
};
\end{lstlisting}
      \end{column}
    \end{columns}
  \end{BoxedExample}

\end{frame}

\begin{frame}[fragile]
  \frametitle{Calling Methods}
  \begin{itemize}
  \item To call methods, you just have to find the pointer in the object
  \item The only things that you need is to provide the \texttt{this} pointer
    to the method.
  \item For cascade of methods call you will need temporary variables to cache
    results of method's calls to avoid multiple calls to the same method.
  \end{itemize}
  \begin{BoxedExample}
    \footnotesize
    \begin{columns}
      \begin{column}{4.5cm}
\begin{lstlisting}
o.set(42);
\end{lstlisting}
      \end{column}
      \begin{column}{4.5cm}
\begin{lstlisting}
o->set(o,42);
\end{lstlisting}
      \end{column}
    \end{columns}
  \end{BoxedExample}

\end{frame}

\begin{frame}[fragile]
  \frametitle{Classes and inheritance}
  \begin{itemize}
  \item Defining classes is straightforward: a classes is a structure together
    with the initialization code to fill the structure.
  \item To avoid unnecessary duplication of pointers to methods, we build an
    external structure with only methods that will be shared by all instances.
  \item Inheritance is also straightforward: structures (state and methods)
    are ordered so that methods described in parent class come first.
  \end{itemize}
  \begin{BoxedExample}
    \footnotesize
\begin{lstlisting}
struct s_obj_vtbl {
  int (*get)(struct s_obj*, void);
  void (*set)(struct s_obj*, int);
};
struct s_obj {
  struct s_obj_vtbl *vtbl;
  int content;
};
\end{lstlisting}
  \end{BoxedExample}
\end{frame}

\pgfdeclareimage[width=8cm]{Vtbl}{Figs/cbangMethodLayout-extract}

\begin{frame}[fragile]
  \frametitle{Classes And Inheritance}
  \begin{center}
    \pgfuseimage{Vtbl}
  \end{center}
\end{frame}

\subsection{Interfaces}

\begin{frame}
  \frametitle{Issues}
  \begin{itemize}
  \item Unlike class inheritance, you can implement several interfaces.
  \item You can inherit from a class and implement interface.
  \item \textbf{$\Rightarrow$ We can't rely on structures layout.}
  \item \emph{What if we use more than one table of methods ?}
  \end{itemize}
\end{frame}

\begin{frame}
  \frametitle{Insufficient information}
  \begin{itemize}
  \item Interfaces are used to pass objects to function without knowing their
    real type.
  \item So, we won't be able to access specific object content that is not
    described in the interface.
  \item So, we need to found a way to access some specific fields of an object
    without knowing the object layout.
  \end{itemize}
\end{frame}

\begin{frame}[fragile]
  \frametitle{Dynamic Dispatcher}
  \begin{itemize}
  \item A possible solution is to have a function that select the right
    table of methods.
  \item This function is just a a big switch using a hash of the interface
    name to select the right table.
  \item Now an object will look like:
  \end{itemize}
  \begin{BoxedExample}
    \footnotesize
\begin{lstlisting}
struct s_obj {
  void *(interface_table)(int);
  struct s_obj_vtbl *vtbl;
  // object content
};
\end{lstlisting}
  \end{BoxedExample}
\end{frame}

\begin{frame}[fragile]
  \frametitle{Avoiding Dispatch Overhead}
  \begin{itemize}
  \item The previous method induces an overhead at each method call (a
    function call)
  \item The idea to avoid this overhead by \emph{transforming} the object when
    we still have the needed information.
  \item Once the object is transformed, we don't need the dynamic dispatcher
    that select the correct method's table.
  \item Let's take a look at the context first:
  \end{itemize}
  \begin{BoxedExample}
    \footnotesize
\begin{lstlisting}
interface A {
  get() : cint;
}
class B support A {
  get() : cint { return 42; }
}
f1(o : A) : cint { return o.get(); }
f2(o : B) : cint { return f1(o); }
\end{lstlisting}

  \end{BoxedExample}
\end{frame}

\subsection{Transforming Objects}

\begin{frame}[fragile]
  \frametitle{Injecting Transformation}
  \begin{itemize}
  \item While there's various ways to implement objects transformation, the
    way we inject information is usually always the same.
  \item The transformation must be injected where we still know the real
    type of the object.
  \end{itemize}
  \begin{BoxedExample}
    The previous function:
    {\footnotesize
\begin{lstlisting}
f2(o : B) : cint { return f1(o); }
\end{lstlisting}
    }

    Into the following:

    {\footnotesize
\begin{lstlisting}
f2(o : B) : cint { return f1( CAST(o,A) ); }
\end{lstlisting}
    }
  \end{BoxedExample}
\end{frame}

\begin{frame}
  \frametitle{How to transform an object}
  \begin{itemize}
  \item There's various way to transform an object:
    \begin{itemize}
    \item You can change the pointer to the table of methods
    \item You can build an helper object around your object
    \item You can have all interfaces in your object and replace the pointer
      of the real object
    \end{itemize}
  \item First two methods has a lot of drawbacks (overhead, inconsistency in
    concurrent context \ldots )
  \item The last method need some refinement
  \end{itemize}
\end{frame}

\begin{frame}[fragile]
  \frametitle{Hosting All Interfaces}
  \begin{itemize}
  \item The idea is to have inside the object a block with, for each supported
    interface, a mini-object with a pointer to corresponding table of methods.
  \end{itemize}
  \begin{BoxedExample}
    \footnotesize
\begin{lstlisting}
struct s_A_vtbl { int (*get)(void); };
struct s_obj_interface_table {
  struct { s_A_vtbl *vtbl; } A;
};
struct s_obj {
  struct s_A_vtbl *interfaces;
  struct s_obj_vtbl *vtbl;
  // object state
  struct s_A_vtbl _interfaces;
};
#define CAST(_OBJ, _INTER_) \
  (&((_OBJ)->interfaces->_INTER_))
\end{lstlisting}
  \end{BoxedExample}
\end{frame}

\begin{frame}[fragile]
  \frametitle{And What About \texttt{this} ?}
  \begin{itemize}
  \item We need to pass a \texttt{this} pointer to our methods.
  \item But the pointer we have in the calling context does not correspond to
    the real object.
  \item So, we provide a pointer to the real object inside the structure
    associated to the interface.
  \end{itemize}
  \begin{BoxedExample}
    \footnotesize
\begin{lstlisting}
struct s_obj_interface_table {
  struct {
    s_A_vtbl *vtbl;
    void     *real_this;
  } A;
};
// Calling a method from interface context
o->vtbl->myMethod(o->real_this);
\end{lstlisting}
  \end{BoxedExample}
\end{frame}

\pgfdeclareimage[width=9cm]{ObjLayout}{Figs/cbangObjecFulltLayout-exp}

\begin{frame}
  \frametitle{Full Object Layout}
  \begin{center}
    \pgfuseimage{ObjLayout}
  \end{center}
\end{frame}

\begin{frame}[fragile]
  \frametitle{Transitive Inheritance}
  \begin{itemize}
  \item When you support an interface that is itself derived from various
    interfaces, you can be transformed to any of these interfaces.
  \item So we need to provide access to these interfaces, but we can be
    transformed in context where we don't know how:
  \end{itemize}
  \begin{BoxedExample}
    \footnotesize
\begin{lstlisting}
interface A { /* Some defs */ }
interface B : A { /* More defs */ }
class C support B { /* implem */ }

f1(o : A) { /* some code */ }
f2(o : B) { f1(o); }
f3(o : C) { f2(o); }
\end{lstlisting}
  \end{BoxedExample}
\end{frame}

\begin{frame}
  \frametitle{Solutions ?}
  \begin{itemize}
  \item Provide a function (similar to the dynamic dispatcher) that provide a
    pointer to the good interfaces (\emph{dynamic cast}.)
  \item Embed recursively interfaces description (probably a lot of
    space overhead.)
  \item We also need conversion for comparison operators.
  \end{itemize}
\end{frame}

\section{Of Structures And Classes}

\begin{frame}
  \frametitle{Why Pointers To Objects ?}
  \begin{itemize}
  \item C++ offers to manipulate object as pointer or directly.
  \item I choose to \textbf{not} follow this approach.
  \item There's almost no reason to transmit objects by copy.
  \item When you need copy, you should obviously do it \textbf{implicitly}.
  \item Copy also introduce new issues: when using an object throught one of
    its parent types or throught an interface, you can't simply rely on C to
    correctly copy the struct since it doesn't know the size.
  \end{itemize}
\end{frame}

\begin{frame}
  \frametitle{Why Do We Have Classes And Structs in \cbang{}}
  \begin{itemize}
  \item You may use structures for other reason than structured data
  \item A good example is when you want a bunch of contiguous heterogeneous
    data
  \item For a lot of cases you don't want meta-data in your structure.
  \item There's also cases where you want data to be copied to called
    functions without overhead.
  \item So, classes and structures serve different purposes and we need both.
  \item If you want \emph{member functions} (or non-virtual methods) acting on
    structures you can use \emph{macro-class} in \cbang
  \end{itemize}
\end{frame}

\begin{frame}
  \frametitle{More on \cbang{}}
  \begin{itemize}
  \item Git repository:\\ {\footnotesize
      \url{http://git.lse.epita.fr/?p=cbang.git}}
  \item Redmine Project:\\ {\footnotesize
      \url{http://redmine.lse.epita.fr/projects/cbang}}
  \item Global Overview Blog's article:\\ {\footnotesize
      \url{http://blog.lse.epita.fr/articles/12-c---system-oriented-programming.html}}
  \item \cbang{} syntax overview:\\ {\footnotesize
      \url{http://blog.lse.epita.fr/articles/23-c---system-oriented-programming---syntax-explanati.html}}
  \end{itemize}
\end{frame}

\pgfdeclareimage[width=8cm]{qmark}{Figs/questionMark}

\begin{frame}
  \frametitle{QUESTIONS ?}
  \begin{center}
    \pgfuseimage{qmark}
  \end{center}
\end{frame}


\end{document}
