%% 20110705-cbang.tex : cbang presentation for LSE Summer week 2011
%% Authors: Marwan Burelle
%% First Created: 2011-06-29
\documentclass[pdftex]{beamer}

\mode<presentation>
{
  \usetheme{LSE}
}
\usepackage{moreverb}
\usepackage[english]{babel}
\usepackage[utf8]{inputenc}
\usepackage{txfonts}
\usepackage{microtype}
\usepackage[T1]{fontenc}
\usepackage{graphicx}
\usepackage{calrsfs}
\usepackage{alltt}
\usepackage{listings}
\usepackage{tikz}
\usetikzlibrary{arrows,snakes,backgrounds,shapes}
\usetikzlibrary{trees}

\title{{C!}}
\subtitle{A System Oriented Programming Language}

\author{Marwan Burelle}

\institute{
  marwan.burelle@lse.epita.fr\\
  \url{http://www.lse.epita.fr/}
}

\date{}

\AtBeginSection[]
{
  \begin{frame}<beamer>
    \frametitle{\insertsection}
    \begin{beamerboxesrounded}{}
      \begin{center}
        \huge{\bf \insertsection}
      \end{center}
    \end{beamerboxesrounded}
  \end{frame}
}

\AtBeginSubsection[]
{
  \begin{frame}<beamer>
    \frametitle{\insertsubsection}
    \begin{beamerboxesrounded}{}
      \begin{center}
        \textbf{\insertsubsection}
      \end{center}
    \end{beamerboxesrounded}
  \end{frame}
}


\definecolorset{cmyk}{}{}%
 {olive,0,0,1,.5}

\setbeamercolor{bExample}{fg=black,bg=olive}
\newenvironment{BoxedExample}
{\begin{beamerboxesrounded}[upper=bExample, shadow=true]{Example:}}
  {\end{beamerboxesrounded}}

\newenvironment{BoxedSyntax}[1]
{\begin{beamerboxesrounded}[shadow=true]{#1}}
  {\end{beamerboxesrounded}}

\newenvironment{BoxedDefinition}[1]
{\begin{beamerboxesrounded}[shadow=true]{#1}}
  {\end{beamerboxesrounded}}

\newcommand{\DefBackButton}[2]{\hfil\hyperlink{#1<#2>}{\beamerreturnbutton{back}}}

\setbeamercolor{CamlRep}{fg=olive}
\newcommand{\CamlRep}[1]{{\usebeamercolor[fg]{CamlRep}\emph{\texttt{#1}}}}

\setbeamercolor{PosYes}{fg=green}
\newcommand{\yes}[1]{{\usebeamercolor[fg]{PosYes} #1}}

\newcommand{\tsetin}[1]{\texttt{IN}(#1)}
\newcommand{\tsetout}[1]{\texttt{OUT}(#1)}
\newcommand{\tclose}[1]{#1_{\mathcal{C}}}
\newcommand{\CS}{{\color{red}CS}}

\begin{document}

\definecolor{orange}{rgb}{1,0.3,0}
\definecolor{kword}{rgb}{0.8,0.5,0}

\lstset{
  language=C,
  basicstyle=\ttfamily,
  commentstyle=\color{orange},
  keywordstyle=\color{kword}\bfseries,
  stringstyle=\color{red},
  showstringspaces=false,
  identifierstyle=,
%  emph={Array,Printf,List,Map,Hashtbl,Sys,Stream,Env},
%  emphstyle=\color{blue}\bfseries,
}

\begin{frame}
  \titlepage
\end{frame}

\begin{frame}<beamer>
  \frametitle{Outline}
  \tableofcontents
\end{frame}



\section{Introduction}

\begin{frame}
  \frametitle{Kernel Programming?}
  \emph{What do we use in kernel programming?}
  \begin{itemize}
  \item Low-level code:
    \begin{itemize}
    \item Direct memory access (reading/writing to in-memory registers
      and other adress-based data access);
    \item Architecture-specific code (ASM inlining);
    \item Low-level execution flow manipulation.
    \end{itemize}
  \item More classical code:
    \begin{itemize}
    \item Data structures;
    \item Algorithms.
    \end{itemize}
  \item Control over binary building.
  \end{itemize}
\end{frame}

\begin{frame}
  \frametitle{Languages for Kernel Programming?}
  \emph{Programming in kernel space requires some features in the
    language and its toolchain.}
  \begin{itemize}
  \item Address manipulation (pointers\ldots);
  \item Function pointers;
  \item Transparent memory mapping of data structures;
  \item Interface with lower-level code (ASM inlining\ldots);
  \item Bitwise operations;
  \item Control over the linking and how the output binary is produced;
  \item Very small or inexistant runtime dependencies.
  \end{itemize}
\end{frame}

\begin{frame}
  \frametitle{Why Only C?}
  \begin{itemize}
  \item Why not a \emph{High Level Language}:
    \begin{itemize}
    \item Most of the modern languages are oriented towards userland
      applications;
    \item \emph{High level languages} (such as Java) often rely on
      specific runtime or worse (Java programs runs on a Virtual
      Machine);
    \item \emph{High level languages} often forbid low level
      manipulations;
    \item Even languages such as C++ or Google's Go requires a
      specific runtime (Go uses a garbage collector, has specific
      calling conventions and various userland only native features)
      incompatible with kernel programming, or at least difficult to
      adapt.
    \end{itemize}
  \item And C?
    \begin{itemize}
    \item C was meant to code kernel!
    \item Produced code is almost a direct translation of the original
      source code;
    \item Most of the current C compilers offer ASM inlining.
    \end{itemize}
  \end{itemize}
\end{frame}

\begin{frame}
  \frametitle{Issues with The C Programming Language}
  \centering{\textbf{\emph{Why would we need another language?}}}

  \ \\

  \begin{itemize}
  \item The official C (ISO) is complex and full of \emph{``things
      that used to be that way and won't change''};
  \item The language has some odd or ambiguous syntax;
  \item The type system of the language is too permissive on some
    subjects while being too restrictive on others;
  \item The language lacks modern programming idioms (like
    objects) that can be useful;
  \item Some features of the language are outdated and do not fit
    well in today's context (\emph{what is the size of an \texttt{int}?
    what really does the keyword \texttt{register}? \ldots}).
  \end{itemize}
\end{frame}

\begin{frame}
  \frametitle{C!}
  \begin{itemize}
  \item {C!} (pronounced \emph{c-bang}) is our attempt to have a new
    programming language for kernel and low-level programming;
  \item Our main goals were:
    \begin{itemize}
    \item Rationalization of the C language syntax;
    \item Minor but useful extensions (mainly syntactic sugar);
    \item Attempting for a better type system (with possibly some
      extensions such as \emph{Generics});
    \item Simple but usable objects;
    \item Language idioms dedicated to kernel/low-level;
      programming
    \item Introducing a module formalism (redundant header files, namespaces, \ldots);
    \item Keeping what makes C a good language for kernel.
    \end{itemize}
  \item Currently C! is more a proof of concept rather than a production
    language, but the compiler prototype can already be used;
  \item Most of our goals are (at least partially) achieved.
  \end{itemize}
\end{frame}



\section{Global Language Overview}

% \begin{frame}
%   \frametitle{Compiler to Compiler}
%   \begin{itemize}
%     \item To fit the language needs (kernel programming, C ABI compatibility,
%       \ldots), there are several options:
%       \begin{itemize}
%       \item write a complete compiler suite: the front-end, the middle-end, and
%         as many back-ends as targetted architectures: too complex and too long
%         for an experimental language;
%       \item yield a front-end for LLVM, for instance: more simple, but still
%         overkill for an experiment;
%       \item make a compiler that outputs C code: enables one to think as ``how
%         it should work'' in C terms, which is quite handy.
%       \end{itemize}
%     \item Of course, the future of the language is not restricted to current
%       choices.
%     \item The current compiler is implemented in OCaml. This compiler is meant
%       to bootstrap a C! compiler in C! (self-hosting).
%   \end{itemize}
% \end{frame}

\begin{frame}
  \frametitle{Compiler to Compiler}
  \begin{itemize}
  \item We chose to use a Compiler to Compiler model;
  \item The C! compiler produces C code that can be compiled using an
    almost standard C compiler (we use C99 and GNU C features);
  \item Why Compiler to Compiler?
    \begin{itemize}
    \item Hard work on native code producing is obtaind for \emph{free};
    \item Most C! specific features are syntaxic sugar over pure C;
    \item Advanced features like object can directly be expressed in C;
    \item The C programming language can be seen as high level portable ASM;
    \item It enforces our idea of a new native general purpose system
      programming language: a strong low level base and non-intrusive
      higher level extensions.
    \end{itemize}
  \end{itemize}
\end{frame}

\begin{frame}
  \frametitle{Syntax Rational}
  \begin{itemize}
  \item The basic idea of C! is to globaly preserve C syntax with rational
    modifications;
  \item The first important point was to design the general syntax in a
    way that can be expressed using a standard parser generator (like
    yacc);
  \item Modifications to the original syntax should be globaly coherent;
  \item The main difference is unambigous \emph{typenames}: type
    identifiers must be identified syntactically: we do not want lexing
    trick!
  \item Modifications on the syntaxic localization of type identifiers
    completely modifies variable and function declarations (and thus
    structures' fields are also modified);
  \item Other syntax differencies are syntaxic sugar and pure C!
    extensions (objects, macro-class, \ldots);
  \end{itemize}
\end{frame}

\begin{frame}[fragile]
  \frametitle{Integer Rational}
  \begin{itemize}
  \item In C integer types are loosely defined, the \texttt{int} type
    obeys two contradictory definitions:
    \begin{itemize}
    \item \texttt{int} type should be of the size of the machine word;
    \item \texttt{int} type should be 2 or 4 bytes (16 or 32 bits) long.
    \end{itemize}
  \item Some others types have a fixed size (\texttt{char},
    \texttt{short}, \ldots) and other are machine dependant
    (\texttt{size\_t} \ldots);
  \item C99 introduces sized types but these definitions are
    \emph{fixed} (\emph{i.e.} the standard defines a fixed set of
    types not a way to define integer type by its size);
  \item We directly introduce size (and signedness) in type
    declaration: a 32 bits signed integer will decalared as
    \verb|int<32>| and the unsigned version as \verb|int<+32>|;
  \item For homogeneity we use a similar scheme for floating point numbers;
  \item Integer with unusal size (less than 64 bits) can also be
    expressed, they will be stored in usual integer and we introduce
    management code to handle overflow.
  \end{itemize}
\end{frame}

\begin{frame}[fragile]
  \frametitle{Minor Extensions}
  \begin{itemize}
  \item Structure definitions are type defitions (no need to use the
    \emph{struct} keywords);
  \item Unsigned integers can be used as bit arrays;
  \item Bitfields are natural extension of sized integer definitions;
  \item Packing of structures is a language feature;
  \item Smarter attempt to handle enumeration's identifier conflict (a
    local variable can't hide an enum value);
  \item Several \emph{work in progress} for eliminating unused (or
    badly named) variable qualifiers;
  \end{itemize}
\end{frame}

\begin{frame}[fragile]
  \frametitle{Objects}
  \begin{itemize}
  \item C! has a simple object oriented syntax extension;
  \item The object extensions is provided for convenience, it is not a
    major paradigm in C!;
  \item One should use objects to simplify data structuring, not code
    structuring;
  \item This extension is build as a syntaxic sugar over
    \emph{hand-made} objects in C;
  \item We have simple inheritance, only virtual methods and method
    overriding (but not overloading);
  \item For unambigous manipulations, objects are always pointers (no
    implicit copy mechanism);
  \item Construction is classical but memory allocation is up to the
    programmer (the constructor take a pointer as first parameter);
  \item Object obey to a single syntax (only \verb|obj.method()| and
    no horrible \verb|obj->method()|);
  \end{itemize}
\end{frame}

\begin{frame}[fragile]
  \frametitle{Modules}
  We introduce a simple \emph{modules model} to avoid naming ambiguity
  in multiple files compilation:
  \begin{itemize}
  \item Symbols and type names are isolated between modules using
    (simplified) namespaces;
  \item Symbols and type names are exported specifiying a namespace:
    \begin{itemize}
    \item \verb#global_variable : int<32>;# won't be exported;
    \item \verb#my_module::global_variable : int<32>;# will be exported.
    \end{itemize}
  \item A module can use another module's exported resources the
    \texttt{import} statement: \verb#import my_module;#;
  \item The backend to C uses mangling to avoid name conflicts between modules
    during the linking;
  \item Unlike in C, a module M2 cannot use a module M3's resources without
    importing it even if both are used by a module M1 (this error is detected
    with splitted compilation).
  \end{itemize}
\end{frame}

\begin{frame}
  \frametitle{Other Extensions}
  \begin{itemize}
  \item C! offers a concept of \emph{macro class}: object oriented
    like syntax for operations call on non-object types;
  \item Various attempts for \emph{external code} inlining (asm or
    pure C) are \emph{(still work in progress)};
  \item Genrated C code is (or we hope it is) supposed to be human-readable and
    directly usable by a C programmer (simpler integration in existing code
    base);
  \item Some newer syntax (still in design phase) will try to solve
    classical low level issue such as automation of register like
    (adress based) data structures.
  \item We also try to have a better type system (there's still some
    issues);
  \end{itemize}
\end{frame}



\section{Syntax and Sugar}

\begin{frame}[fragile]
  \frametitle{Syntax Rational: type position}
  \begin{itemize}
  \item The main C's grammar issue is about type position:
    \begin{center}\scriptsize
      \verb#struct my_struct[5] *my_var;#
      \hspace{0.3cm}$\rightarrow$\hspace{0.3cm}
      \verb#my_var : my_struct[5] *;#
    \end{center}
  \item The former style forces to hack the lexer in order to memorize typedefs. The
    last one introduces no ambiguity.
  \item This syntax is used everytime one associates a type with a name:
    \begin{itemize}
      \item Parameters and return type in function declarations:
        \begin{center}\scriptsize
          \verb#double my_function(long prm1, float prm2);# \\
          \hspace{0.3cm}$\downarrow$\hspace{0.3cm} \\
          \verb#my_function(prm1 : int<32>, prm2: float<32>) : float<64>;# \\
        \end{center}
      \item Field declaration for boxed types (structures, classes, \ldots):
        \begin{center}\scriptsize
          \verb#struct my_struct { long field; … };# \\
          \hspace{0.3cm}$\downarrow$\hspace{0.3cm} \\
          \verb#struct my_struct { field : int<32>; … }# \\
        \end{center}
    \end{itemize}
  \end{itemize}
\end{frame}

\begin{frame}[fragile]
  \frametitle{Syntax Rational: pointer to function}

  \begin{itemize}
  \item Pointer to function types are made more readable:
    \begin{center}\scriptsize
      \verb#int (*my_func)(int, char *)# \\
      \hspace{0.3cm}$\downarrow$\hspace{0.3cm} \\
      \verb#my_func : <(int<32>, int<8> *)> : int#
    \end{center}
  \item Once more, the trick is to isolate the type and the identifier.
  \end{itemize}
\end{frame}

\begin{frame}[fragile]
  \frametitle{Syntax Rational: casting}

  \begin{itemize}
  \item In C, can you determine if \verb#(x)(y)# is:
    \begin{itemize}
      \item a cast of \texttt{y} to the \texttt{x} type, or…
      \item a call of the function \texttt{x} with \texttt{y} as an argument?
    \end{itemize}
  \item Casts in C! obey the unambigous type identifier rules!
    \begin{center}\scriptsize
      \verb#string = (char *) my_ptr;# \\
      \hspace{0.3cm}$\downarrow$\hspace{0.3cm} \\
      \verb#string = (my_ptr : int<8> *);#
    \end{center}
  \item This syntax is coherent with previous changes.
  \end{itemize}
\end{frame}

\begin{frame}[fragile]
  \frametitle{Syntax Rational: minor changes}

  \begin{itemize}
  \item Trailing \verb|';'| at the end of structure and class declarations is
    no longer needed;
  \item Function calls have a single syntax: a functional expression
    (direct identifier, function pointer, array or structure element)
    is applied to a vector of arguments;
    % TODO: what is the other one in C?
  \end{itemize}
\end{frame}

\begin{frame}[fragile]
  \frametitle{Syntactic sugar: bit access}

  \begin{itemize}
  \item System code often deals with bit fields;
  \item One can handle it playing with bitwise operations (masks, shifts,
    \ldots); \item Or using bit fields in packed C's structures;
  \item C! brings another possibility: use unsigned integer as a bit array:
\begin{verbatim}
switch_feature(flags : int<+32> *) : int<+32>
{
    (*flags)[3] = !(*flags)[3];
    return (*flags)[3];
}
\end{verbatim}
  \end{itemize}
\end{frame}



\section{Objects in C!}



\begin{frame}
  \frametitle{Object Model}
  \begin{itemize}
  \item Object in C! is a class-based OOP model
  \item Intuitively, we design objects in C and then provide a
    syntaxic sugar for C!
  \item We choose to keep the object model as simple as we can
  \item Basicaly we have:
    \begin{itemize}
    \item Usual class definintions
    \item All object's code is in class definintion
    \item Methods are \emph{virtual} (in C++ terminology)
    \item There's no visibility control (everything is \emph{public})
    \item Methods can be overriden but not overloaded
    \item Objects creation is divided in two parts: allocation
      (delegated to the programmer) and object initialization
    \end{itemize}
  \item We try to avoid most C++ annoying aspects: no implicit copy,
    no syntax hell (object, reference, pointer ... ), no syntax noise.
  \end{itemize}
\end{frame}

\begin{frame}[fragile]
  \frametitle{Inner Representation}
  \begin{itemize}
  \item In C! objects are structures with special inner structure
    containning function pointers:
\begin{lstlisting}
struct my_object {
  // vtable
  struct my_object_methods *_methods;
  // attributes
  int x,y;
};

struct my_object_methods {
  int (*getX)(void);
  int (*getY)(void);
};
\end{lstlisting}

  \end{itemize}
\end{frame}

\begin{frame}[fragile,fragile]
  \frametitle{Objects Syntax}
\begin{verbatim}
class A {
  x : int<32>;
  get() : int<32> { return x; }
  set(_x : int<32>) : void { x = _x; }
  double() : void
  {
    this.set(2 * this.get());
  }
}

class B {
  x : A;
  get() : A { return x; }
}
\end{verbatim}
\end{frame}

\begin{frame}[fragile]
  \frametitle{Objects Syntax}
  \begin{itemize}
  \item The syntax embeds all the cooking for objects manipulation in
    C:
    \begin{itemize}
    \item Method call such as \verb|a.get()| will be rewritten as
      \verb|a->_methods->get(a)|
    \item Cascading methods call is managed using compound expressions
      and temporary local variables: the call
      \verb|b.get().set(42)| will be translated as:
\begin{lstlisting}
({
   A _cbtmp1 = b->_methods->get(b);
   _cbtmp1->_methods->set(_cbtmp1, 42) ;
})
\end{lstlisting}
    \item Actual constructors are build using class variables (class
      definition can take parameters) visible only at creation time.
    \item The constructor always take as first a pointer to a suitable
      memory location to store the object (of type \verb|void*| to
      avoid typing issues.)
    \end{itemize}

  \end{itemize}
\end{frame}

\section{The Macro Class Concept}

\begin{frame}
  \frametitle{Macro Class ?}
  \begin{itemize}
  \item The idea is to provide an object like syntax to manipulate
    non-object data.
  \item When defining a macro-class we give a base type (the storage
    type) and a set of \emph{operations}.
  \item A fake \emph{this} is available in operations code, it
    represents the inner value
  \item operations can be \emph{const} (no modifications to the inner
    value) or not.
  \item Normally methods calls on macro-class are replace by C macro.
  \item Macro-class are a simple syntax extensions to handle typed
    macro over specific data.
  \end{itemize}
\end{frame}

\begin{frame}
  \frametitle{Macro class usage}
  \begin{itemize}
  \item Macro class can be used where object-oriented syntax are
    convenien but full-object implementation is useless.
  \item A classical example is evolved integer flags such as status
    return of a \texttt{wait} syscall. One can define setting and
    testing of various properties of the status as method call that in
    fact will implement the usual macro.
  \item The purpose is to provide convenient syntax for low level
    behaviour code (for exemple, it will possible to include assembly
    code in a macro operations) without any overhead or runtime code.
  \end{itemize}
\end{frame}

\begin{frame}[fragile,fragile]
  \frametitle{Example}
  \footnotesize
\begin{verbatim}
macro class MC : int<+32>
{
  get() const : int<+32>
  {
    return (this * 2);
  }

  set(x : int<+32>) : void
  {
    this = x/2;
  }
}

f() : void
{
  x : MC;
  // x is really an int<+32>
  x = 0;
  // but it can be used like an object
  x.set(42);
}
\end{verbatim}
\end{frame}

\section{State of The Prototype}

\begin{frame}
  \frametitle{State of The Prototype}

  \pause

  \begin{itemize}
    \item Lexing/Parsing : check; \pause
    \item Type checking : check, even if a few improvements are scheduled;
      \pause
    \item Module support: very partial… but coming soon! (the identifiers
    mangling is still to be decided) \pause
    \item Object support: check; \pause
    \item Macro class support: partial; \pause
    \item Stay tuned!
  \end{itemize}
\end{frame}



\section{Future Directions}

\begin{frame}
  \frametitle{Work in progress}
  \begin{itemize}
  \item Finish and fix work in the compiler
  \item Fix syntax for inlining
  \item Fix the object constructor troll
  \item Solve various typing issues
  \item Find a better way to implement class macro ops (inlined
    functions ?)
  \item Have a tool chain ...
  \end{itemize}
\end{frame}

\begin{frame}
  \frametitle{Future Works ?}
  \begin{itemize}
  \item Data mapping: automatic access to register like data
  \item Method Overloading
  \item Plugin and annotation (external processing of code)
  \item Generics and/or Template
  \item Global static constructions
  \item Native compiler (does-it make sense ?)
  \item Self-host (C! in C!)
  \item Front-end integration in gcc/clang
  \item Code validation and static checking
  \end{itemize}
\end{frame}

\begin{frame}
  \frametitle{QUESTIONS ?}
  \begin{center}
    \includegraphics[width=8cm]{Figs/questionMark}
  \end{center}
\end{frame}

\end{document}
